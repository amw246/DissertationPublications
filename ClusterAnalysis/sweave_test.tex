\documentclass{article}\usepackage[]{graphicx}\usepackage[]{color}
%% maxwidth is the original width if it is less than linewidth
%% otherwise use linewidth (to make sure the graphics do not exceed the margin)
\makeatletter
\def\maxwidth{ %
  \ifdim\Gin@nat@width>\linewidth
    \linewidth
  \else
    \Gin@nat@width
  \fi
}
\makeatother

\definecolor{fgcolor}{rgb}{0.345, 0.345, 0.345}
\newcommand{\hlnum}[1]{\textcolor[rgb]{0.686,0.059,0.569}{#1}}%
\newcommand{\hlstr}[1]{\textcolor[rgb]{0.192,0.494,0.8}{#1}}%
\newcommand{\hlcom}[1]{\textcolor[rgb]{0.678,0.584,0.686}{\textit{#1}}}%
\newcommand{\hlopt}[1]{\textcolor[rgb]{0,0,0}{#1}}%
\newcommand{\hlstd}[1]{\textcolor[rgb]{0.345,0.345,0.345}{#1}}%
\newcommand{\hlkwa}[1]{\textcolor[rgb]{0.161,0.373,0.58}{\textbf{#1}}}%
\newcommand{\hlkwb}[1]{\textcolor[rgb]{0.69,0.353,0.396}{#1}}%
\newcommand{\hlkwc}[1]{\textcolor[rgb]{0.333,0.667,0.333}{#1}}%
\newcommand{\hlkwd}[1]{\textcolor[rgb]{0.737,0.353,0.396}{\textbf{#1}}}%

\usepackage{framed}
\makeatletter
\newenvironment{kframe}{%
 \def\at@end@of@kframe{}%
 \ifinner\ifhmode%
  \def\at@end@of@kframe{\end{minipage}}%
  \begin{minipage}{\columnwidth}%
 \fi\fi%
 \def\FrameCommand##1{\hskip\@totalleftmargin \hskip-\fboxsep
 \colorbox{shadecolor}{##1}\hskip-\fboxsep
     % There is no \\@totalrightmargin, so:
     \hskip-\linewidth \hskip-\@totalleftmargin \hskip\columnwidth}%
 \MakeFramed {\advance\hsize-\width
   \@totalleftmargin\z@ \linewidth\hsize
   \@setminipage}}%
 {\par\unskip\endMakeFramed%
 \at@end@of@kframe}
\makeatother

\definecolor{shadecolor}{rgb}{.97, .97, .97}
\definecolor{messagecolor}{rgb}{0, 0, 0}
\definecolor{warningcolor}{rgb}{1, 0, 1}
\definecolor{errorcolor}{rgb}{1, 0, 0}
\newenvironment{knitrout}{}{} % an empty environment to be redefined in TeX

\usepackage{alltt}
\usepackage{subcaption}
\usepackage{booktabs}
% \setkeys{Gin}{width=1\textwidth}
% \setkeys{Gin}{height=1\textheight}
\IfFileExists{upquote.sty}{\usepackage{upquote}}{}
\begin{document}

\begin{table}[ht]
\centering
\caption{Degree Band Descriptives} 
\begin{tabular}{lccccc}
  \toprule
 & BA1 & BA2 & AA1 & AA2 & Total \\ 
  \midrule
White & 34.49 & 26.33 & 19.67 & 24.38 & 25.18 \\ 
  Black & 22.38 & 27.41 & 34.78 & 34.71 & 31.08 \\ 
  Hispanic & 23.56 & 30.87 & 34.82 & 27.43 & 29.72 \\ 
  Asian & 19.57 & 15.40 & 10.72 & 13.48 & 14.02 \\ 
  Male & 39.94 & 42.66 & 46.18 & 38.91 & 42.51 \\ 
  Female & 60.06 & 57.34 & 53.82 & 61.09 & 57.49 \\ 
  Independent & 6.51 & 6.01 & 30.00 & 20.27 & 19.99 \\ 
  Dependent & 93.49 & 93.99 & 70.00 & 79.73 & 80.01 \\ 
  Not Pell Recipient & 49.09 & 43.29 & 43.18 & 41.53 & 44.41 \\ 
  Pell Recipient & 50.91 & 56.71 & 56.82 & 58.47 & 55.59 \\ 
  Delayed Entry & 14.76 & 11.65 & 41.61 & 31.79 & 30.48 \\ 
  No Delay in Entry & 85.24 & 88.35 & 58.39 & 68.21 & 69.52 \\ 
  Fall & 91.10 & 90.73 & 72.02 & 77.10 & 79.19 \\ 
  Spring & 8.90 & 9.27 & 27.98 & 22.90 & 20.81 \\ 
  Age at Entry & 19.28 & 18.97 & 22.44 & 20.94 & 21.07 \\ 
  College Prep Units & 18.20 & 17.01 & 10.76 & 13.09 & 13.73 \\ 
  HS GPA & 82.11 & 79.03 & 73.11 & 74.83 & 76.51 \\ 
  SAT Total before Transformation & 972.54 & 924.15 & 785.34 & 808.33 & 885.54 \\ 
  First Sem. Credits & 10.46 & 7.65 & 4.55 & 6.96 & 6.89 \\ 
  First Sem. GPA & 2.64 & 1.91 & 2.06 & 2.58 & 2.36 \\ 
   \bottomrule
\end{tabular}
\end{table}





Ok. I finally got this to work. But it involves running the ENTIRE data set up 
file because objects in the global enviroment are not in the scope of the local
environment of the knitr R code chunk (unknown if multiple chunks in the same 
knitr document have the same scope) for reproducibility purposes. This works, but 
it is EXTREMELY inefficient. The solution may be to save the data to an Rdata file
and read that. 

Done. It worked SO much faster. , out.extra='angle=90' 
% 
% \begin{figure}{t}
% \centering
% 
% <<echo=FALSE, message = FALSE, out.extra='angle=90'>>=
% library(TraMineR)
% load("/Volumes/untitled/PTC.BA1.RData")
% load("/Volumes/untitled/PTC.BA2.RData")
% load("/Volumes/untitled/PTC.AA1.RData")
% load("/Volumes/untitled/PTC.AA2.RData")
% source("/Users/andrewwallace/DissertationPublications/CreateStateSeqObj.R")
% source("/Users/andrewwallace/DissertationPublications/CreateTopTwentyFiguresBA1.R")
% @
%   \caption{Top Twenty Sequences, BA1}
%   \label{fig:Top20BA1}
%   \end{figure}
%   
% 	\begin{figure}[t]
%   \centering
% <<echo=FALSE, message = FALSE>>=
% library(TraMineR)
% load("/Volumes/untitled/PTC.BA1.RData")
% load("/Volumes/untitled/PTC.BA2.RData")
% load("/Volumes/untitled/PTC.AA1.RData")
% load("/Volumes/untitled/PTC.AA2.RData")
% source("/Users/andrewwallace/DissertationPublications/CreateStateSeqObj.R")
% source("/Users/andrewwallace/DissertationPublications/CreateTopTwentyFiguresBA2.R")
% @
% 		\caption{BA2}
%     \caption{Top Twenty Sequences, BA2}
% 	\end{figure}
% 
% 
% 
% \newpage
% 
% Why is this not working
% 
% 
\end{document}
